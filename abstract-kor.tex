\clearpage
\begin{flushleft}
    \Large{
        \textbf{국 문 초 록}\\
    }
\vspace{\baselineskip}
\end{flushleft}

\begin{centering}
    \LARGE{
        \textbf{딥러닝 및 생성 인공지능 모델을 활용한 동적 스펙트럼 공유} \\
    }
\vspace{\baselineskip}
\end{centering}


\begin{flushright}
    Gamage Amanda Sheron \\
    전기전자공학부 \\
    연세대학교 일반대학원
\end{flushright}

동적 스펙트럼 공유는 무선 통신 자원의 효율적인 활용을 위한 핵심 요소이며, 특히 Citizens Broadband Radio Service (CBRS) 대역에서 중요한 역할을 합니다. 
본 학위논문에서는 생성 모델링, 지도 학습 기반 분류, 그리고 심층 강화학습(DRL)을 통합한 지능형 스펙트럼 액세스를 위한 새로운 종단 간 AI 기반 프레임워크를 제안합니다. 
제안된 시스템은 전통적인 스펙트럼 액세스 시스템(SAS)을 대체할 수 있는 학습 기반 대안으로, 실시간 스펙트럼 점유 탐지 및 자율 채널 선택을 수행할 수 있도록 설계되었습니다.

학습을 위한 레이블링 데이터의 부족 문제를 해결하기 위해, 본 논문에서는 세 가지 종류의 생성 AI 모델—생성적 적대 신경망(GAN), 벡터 양자화 변분 오토인코더(VQ-VAE), 
잡음 제거 확률 확산 모델(DDPM)—을 활용하여 간섭 및 충돌 사례를 포함한 다양한 CBRS 시나리오를 나타내는 고품질 스펙트로그램 이미지를 생성합니다. 이렇게 생성된 합성 
스펙트로그램은 충돌 이벤트의 이진 분류를 위한 합성곱 신경망(CNN) 학습에 사용됩니다. 실험 결과에 따르면, 시각적 품질과 분류 정확도 모두에서 DDPM이 가장 우수한 성능을 
보이며, GAN 및 VQ-VAE를 크게 상회하는 결과를 보여주었습니다.

MATLAB 기반의 CBRS 시뮬레이션 환경이 구축되었으며, 여기서 심층 Q-네트워크(DQN)를 사용하여 학습된 DRL 에이전트가 스펙트럼 환경과 상호작용합니다. 에이전트는 
CNN으로부터 채널 점유 예측을 받아 Incumbent 사용자 및 우선 접근 면허(PAL) 사용자와의 충돌을 피하면서 일반 허가 접근(GAA) 채널에서의 처리량을 극대화하도록 학습합니다. 
정량적 결과에 따르면, DDPM으로 생성된 데이터를 활용해 학습된 DQN 에이전트는 에피소드당 평균 보상 79.65와 충돌 0건이라는 안전하고 효과적인 채널 선택 성능을 보였습니다. 
반면, GAN 기반 에이전트는 더 높은 평균 보상(86.78)을 달성했지만 에피소드당 평균 충돌이 0.74건 발생했으며, VQ-VAE 및 실제 데이터 기반 에이전트는 각각 12.6건 및 17.44건의 
높은 충돌률을 보이며 보상 또한 낮거나 유사한 수준에 머물렀습니다. 이러한 결과는 생성 모델의 품질이 스펙트럼 액세스에서의 성능과 운영 안전성을 모두 보장하는 데 있어 중요함을 보여줍니다.

본 논문은 생성 AI, 심층 분류, 그리고 DRL의 통합이 CBRS 대역에서의 스펙트럼 공유 문제에 대해 확장 가능하고 지능적인 해결책을 제공함을 결론짓습니다. 제안된 모듈형 아키텍처, 
고성능 학습 모델, 그리고 시뮬레이션 결과는 향후 멀티채널 확장, 실시간 배포, 비지도 학습 기반 스펙트럼 관리 기법 등의 연구 확장을 위한 기반을 마련합니다.

\blfootnote{핵심어: Citizens Broadband Radio Service (CBRS), 생성적 적대 신경망 (GAN), 벡터 양자화 변분 오토인코더 (VQ-VAE), 잡음 제거 확산 확률 모델 (DDPM), 
심층 Q-네트워크 (DQN), 스펙트럼 액세스 시스템 (SAS), 동적 스펙트럼 공유, 강화 학습, 합성곱 신경망 (CNN)}
