\chapter{Conclusion}
\label{chap:conclusion}

This thesis presented a comprehensive study on dynamic spectrum sharing in the Citizens Broadband Radio Service (CBRS) environment through the integration of deep learning and generative artificial intelligence models. The growing complexity and demand for wireless communication systems necessitate more intelligent, flexible, and scalable approaches to spectrum management. Traditional static and rule-based systems fail to accommodate the rapid evolution of wireless technologies and user behavior, especially in spectrum-constrained environments like CBRS. In response, we proposed a novel pipeline that synergizes generative models, convolutional neural networks (CNN), and deep reinforcement learning (DRL) to enable dynamic and efficient channel allocation.

A key innovation of this work lies in the use of generative models—GAN, VQ-VAE, and DDPM—to synthesize realistic spectrogram images simulating various channel occupancy scenarios. These include non-interfering states (empty, radar, LTE) and critical interference cases (collision). The synthetic spectrograms were then used to train a CNN-based classifier for automatic collision detection, a critical functionality in any spectrum access system. Our comparative study revealed that DDPM outperformed both GAN and VQ-VAE in terms of Fréchet Inception Distance (FID) and classification accuracy, indicating superior image quality and diversity. These findings align with the emerging consensus in the machine learning community regarding the superiority of diffusion-based models for high-fidelity generation tasks.

To evaluate the practical utility of the generative models and the trained CNN, we implemented a MATLAB-based CBRS simulation environment. In this simulated environment, a DRL agent was tasked with making real-time channel allocation decisions based on the CNN’s predictions. This framework effectively replaces a conventional Spectrum Access System (SAS), offering a decentralized and learning-based alternative that can adapt to previously unseen scenarios. The DRL agent was trained using a Deep Q-Network (DQN) architecture with experience replay and Double DQN to ensure stability. By using a binary channel occupancy vector derived from spectrogram classification, the agent learned to avoid collisions with Incumbent and PAL users while maximizing spectral efficiency for General Authorized Access (GAA) users.

This work not only demonstrates the viability of using AI-driven techniques for spectrum sharing but also provides evidence of their practical superiority over traditional hand-engineered solutions. By generating diverse training data, we overcame one of the primary challenges in supervised learning: the scarcity of labeled samples in complex environments. The trained CNN achieved high accuracy even for subtle or rare collision patterns, and the DRL agent showed robust behavior during spectrum reallocation under dynamically changing conditions.

Furthermore, we demonstrated that combining synthetic data with real training data can enhance classifier generalization. This is particularly important for regulatory and commercial deployment, where real-world datasets may be highly imbalanced or incomplete. The DDPM-generated spectrograms not only improved classification performance but also contributed to a more diverse and representative training set, thereby reducing model bias and improving resilience in unseen environments.

Despite these achievements, there are several limitations and open problems that warrant further research. First, while our models distinguish between classes like radar and LTE effectively, they are less adept at distinguishing between PAL and GAA users due to similar waveform characteristics. This is a significant limitation in a practical SAS system, where different policy constraints apply to each user type. Future work could address this challenge by incorporating metadata or control-plane information, such as the Mobile Network Code (MNC) obtained via System Information Block (SIB1) messages, into the decision-making pipeline.

Another future direction involves improving the temporal and frequency resolution of the spectrograms. Currently, we use fixed STFT parameters to generate 64x64 or 224x224 resolution spectrograms, which may limit the fidelity of subtle spectral features. Adaptive time-frequency representations, or even raw I/Q signal modeling using 1D convolutional networks or transformers, may offer better performance for dynamic and complex signals. Furthermore, while this work focused on two channels, scaling to multi-channel (N > 2) spectrum environments introduces new challenges in DRL action space, spectrogram representation, and interference modeling.

Expanding the generative models themselves is also a promising path. While conditional DDPM performed best among the tested models, hybrid approaches such as classifier-guided diffusion, or Score-based Generative Models, could yield even better results. Similarly, self-supervised learning or contrastive learning approaches could reduce the reliance on class labels, enabling spectrum analysis in unsupervised or weakly labeled domains.

From a systems perspective, this research opens the door to real-time, learning-driven SAS implementations. By integrating tools such as Keysight Nemo or open-source LTE/5G sniffers, the DRL agent could be trained or validated in near-real-world conditions. The modular nature of the system separating generation, classification, and control—makes it easy to swap components or upgrade specific functionalities as new models and techniques emerge. In practical deployments, this could enhance spectrum efficiency, reduce operational costs, and accelerate the adoption of shared spectrum systems globally.

In conclusion, this thesis has shown that generative AI models, when combined with CNN-based classification and DRL-based control, offer a powerful and scalable framework for dynamic spectrum sharing in CBRS. The proposed system bridges the gap between data scarcity and real-time decision-making, and it sets a foundation for future research in AI-driven spectrum management. Our experiments demonstrate not only the technical feasibility of such a system but also its strong performance across key metrics of accuracy, diversity, and adaptability. As spectrum becomes an increasingly scarce and valuable resource, AI-based systems like the one proposed here will be essential to ensuring equitable, efficient, and intelligent spectrum usage in the wireless networks of the future.