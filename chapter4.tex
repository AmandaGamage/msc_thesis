\chapter{Discussion}
\label{chap:discussion}

In this paper, we presented a deep learning-based system for interference classification in the \gls{cbrs} spectrum, leveraging synthetic data generated by three distinct generative models: \gls{gan}s,\gls{ddpm}s, and \gls{vq-vae}s. A \gls{cnn} was trained using the datasets produced by each generative model, and the classification accuracies were compared. Then that pre-trained \gls{cnn} was used to detect Incumbent users in a simulated CBRS environment with a \gls{drl} agent for channel allocation. The performance of the \gls{drl} agent and the \gls{cnn} was measured Additionally, the \gls{fid} was used to evaluate the quality and diversity of the images generated by the models.  

The experimental results show that the images generated by \gls{ddpm}, \gls{gan}, and \gls{vq-vae} achieved classification accuracies of 81.9\%, 78.5\%, and 71.3\%, respectively. Among the three models, \gls{ddpm} achieved the lowest \gls{fid} score, indicating that it produced images with the highest diversity while maintaining fidelity. These findings suggest that \gls{ddpm}-generated datasets when combined with \gls{cnn}-based classification, provide the most effective approach for interference classification in the \gls{cbrs} spectrum. The combination of superior classification accuracy and low \gls{fid} score underscores the advantage of \gls{ddpm} in this application.  

Furthermore, the performance of \gls{dqn} agents trained using each generative model was analyzed in terms of reward and safety in a simulated CBRS environment. The \gls{ddpm}-trained agent showed the best safety profile, maintaining a mean reward of 79.65 with zero collisions per episode, demonstrating strong generalization and conservative channel selection. The \gls{gan}-trained agent achieved the highest mean reward of 86.78, though it incurred an average of 0.74 collisions per episode, reflecting a more aggressive but risk-prone strategy. In contrast, the \gls{vq-vae}-trained agent had a lower mean reward of 77.62 and suffered from 12.6 collisions per episode, while the agent trained on the original dataset yielded a reward of 78.4 with 17.44 collisions per episode. These results highlight the effectiveness of \gls{ddpm} in producing not only high-quality training data but also enabling safe and efficient decision-making by reinforcement learning agents in dynamic spectrum-sharing settings.

While the proposed system is tailored for collision detection and spectrum management within \gls{cbrs}, its architecture and methodology are inherently adaptable to other dynamic spectrum-sharing frameworks. The use of generative models such as \gls{ddpm}s, \gls{gan}s, and \gls{vq-vae}s for creating synthetic collision scenarios and training robust classifiers is not limited to \gls{cbrs} but can be extended to any system that requires dynamic interference management and real-time decision-making.  

For instance, \gls{tvws}\cite{13}, which allow unlicensed devices to operate in unused portions of the broadcast television spectrum, face challenges similar to \gls{cbrs} in managing interference between primary and secondary users. The synthetic data generation techniques presented in this work can aid in training deep learning classifiers for detecting spectrum misuse or interference in such systems.  

Additionally, this approach is well-suited for 5G and 6G networks, where ultra-dense deployments and diverse spectrum use cases demand advanced interference detection and management capabilities. In these environments, generative models can simulate interference scenarios across various network slices, enabling operators to develop more resilient \gls{dsa} systems.  

An important future research direction is addressing the challenge of distinguishing between \gls{pal} and \gls{gaa} users, particularly in scenarios where their spectral behavior appears identical. One promising approach could involve incorporating network-level information, such as decoding the System Information Block (SIB1)  message to retrieve the \gls{mnc} of colliding networks. This would allow for identifying the specific networks involved in suspected \gls{pal} and \gls{gaa} collisions. Integrating tools such as Keysight Nemo with the proposed deep learning framework could create a hybrid system where spectral analysis detects potential collisions and network monitoring tools provide additional validation and identification. This complementary methodology offers a pathway for further enhancing spectrum management and interference classification strategies. 

